\documentclass{article}
\usepackage[utf8]{inputenc}
\usepackage[margin=1in]{geometry}
\usepackage{xcolor}
\usepackage{changepage}
\usepackage{titling}
\usepackage{amsmath}
\usepackage{amsfonts}
\usepackage{array}
\droptitle = -1in

% For setting page numbering, but you've already disabled it.
\setlength{\parindent}{0cm}
\pagenumbering{gobble}

\title{Homework 1}
\author{Seven Lewis}
\date{4/12/24}


% Defining custom commands for colored text
\newcommand{\problem}[1]{\noindent\textcolor{black}{#1}}
\newcommand{\solution}[1]{\noindent\textcolor{red}{#1}}

% Alternatively, defining environments for problems and solutions
\newenvironment{Problem}
{\noindent\color{black}}
{\newline}

\newenvironment{Solution}
{\noindent\color{red}}
{\newline}

\begin{document}

\maketitle

% Using custom commands for inline coloring
\section*{1. True or False (10pts)}
\problem{1)	Every element of a subset is also an element of the original set.}

\solution{True}


\problem{2) The statement $(p \rightarrow q)$ is false only when $p$ is true, and $q$ is false.}

\solution{True}


\problem{3) The empty set is a subset of every set.}

\solution{True}


\problem{4) If $A \subseteq B$ and $B \subseteq A$, then $A$ and $B$ are equal.}

\solution{True}
 

\problem{5) The negation of the statement $(p \lor q)$ is logically equivalent to $(\sim p \land \sim q)$}

\solution{True}


\problem{6) The intersection of two sets is always non-empty.}

\solution{False}


\problem{7) If sets $A$ and $B$ are disjoint, they have no elements in common.}

\solution{True}


\problem{8) The converse of the statement $(p \rightarrow q)$ is logically equivalent to $(q \rightarrow p)$.}

\solution{True}

\problem{9) The union of a set and its complement is the universal set.}

\solution{True}

\problem{10) Here is a list data structure: $[1,2,2,3]$. When you convert this list to a set, the cardinality of that set is $4$.}

\solution{False}



\vspace*{27em}


\section*{2. }

\begin{Problem}
    (5pts) A survey was taken by 200 students on what courses they have taken. The
    results show that 90 students took CS, 110 took Mathematics, 60 took
    Physics, 20 took both CS and Math, 20 took both CS and Physics, and 30 took
    both Math and Physics. How many students have taken course in all three
    areas, including Mathematics, CS, and Physics? Please demonstrate your
    stepwise analysis.
\end{Problem}


\begin{Solution}
    Let the set $U$ be the set of students who took the survey, $C$ be the set of students who have taken a course in computer science,
    similarly let $M$ be the set for math,  $P$ for physics. 

    \phantom{ }

    Given: 
    \begin{adjustwidth}{2em}{0em} 
        $|U| = 200$

        $|C \cup M \cup P| = 200$ 

        $|C| = 90$

        $|M| = 110$

        $|P| = 60$

        $|C \cap M| = 20$

        $|C \cap P| = 20$
        
        $|M \cap P| = 30$ 
    \end{adjustwidth}

    \phantom{ }

    Find: $|C \cap M \cap P| = x$

    \phantom{ }

    First let's isolates the different sections of the ven-diagram:

    \begin{adjustwidth}{2em}{0em}
        Students with all three classes:
        \begin{adjustwidth}{2em}{0em}
            $|C \cap M \cap P| = x$
        \end{adjustwidth}

        Students with only two classes:

        \begin{adjustwidth}{2em}{0em}
            $|(C \cap M) - (C \cap M \cap P)| = 20 - x$

            $|(C \cap P) - (C \cap M \cap P)| = 20 - x$

            $|(M \cap P)- (C \cap M \cap P)| = 30 - x$ 
        \end{adjustwidth}

        Students with only one class:

        \begin{adjustwidth}{2em}{0em}
            $|C - (M \cup P)|  = |C| - |(C \cap M) - (C \cap M \cap P)| - |(C \cap P) - (C \cap M \cap P)| - |C \cap M \cap P|$

            \hspace*{6.585em}$ = 90 - (20 - x) - (20 - x) - x = 50 + x$

            $|M - (C \cup P)| = |M| - |(C \cap M) - (C \cap M \cap P)| - |(M \cap P)- (C \cap M \cap P)| - |C \cap M \cap P|$

            \hspace*{6.585em}$ = 110 - (20 - x) - (30 - x) - x = 60 + x$

            $|P - (C \cup M)| = |P| - |(C \cap P) - (C \cap M \cap P)| - |(M \cap P)- (C \cap M \cap P)| - |C \cap M \cap P|$

            \hspace*{6.585em}$ = 60 - (20 - x) - (30 - x) - x = 10 + x$
        \end{adjustwidth}
    \end{adjustwidth}

    \ \\

    Now let's add all the disjoint sets together:

    $|C \cup M \cup P| = x + (20 - x) + (20 - x) + (30 - x) + (50 + x) + (60 + x) + (10 + x)$

    \phantom{ }

    \hspace*{5.71em}$= 190 + x = |U| = 200$

    \phantom{ }

    \hspace*{5.71em}$x = 10$

    \phantom{ }

    Therefore, $10$ students have taken all three courses. 
\end{Solution}







\vspace*{5em}




\section*{3. }

\begin{Problem}
    (5pts) A detective has interviewed four witnesses of a crime. From the stories of the witnesses the detective has concluded that:
    \begin{enumerate}
        \item If the butler is lying, then so is the handy man.
        \item The cook and the butler cannot both be telling the truth.
        \item The cook and the gardener are not both lying.
        \item If the butler is telling the truth, then the gardner is lying. 
    \end{enumerate}

    For each of the four witnesses, can the detective determine whether that person is telling the truth or lying? Use the truth table to explain your reasoning. Please use the following notations in your reasoning –
    
    \begin{adjustwidth}{5em}{0em}
        $B$: The butler is telling the truth.

        $C$: The cooking is telling the truth.

        $G$: The gardener is telling the truth.

        $H$: The handyman is telling the truth.
    \end{adjustwidth}

    \phantom{ }
\end{Problem}

\begin{Solution}
    Given (in symbols):
    \begin{adjustwidth}{2em}{0em}
        $\sim B \rightarrow \sim H$

        $\sim (C \land B)$

        $\sim (\sim C \land \sim G)$

        $B \rightarrow \sim G$
    \end{adjustwidth}

    \phantom{ }

    Derived:

    \begin{adjustwidth}{2em}{0em}
        $H \rightarrow B$

        $(\sim C \lor \sim B) \equiv (C \rightarrow \sim B) \equiv (B \rightarrow \sim C)$

        $(C \lor G) \equiv (\sim \sim C \lor G) \equiv (\sim C \rightarrow G)$

        $G \rightarrow \sim B$
    \end{adjustwidth}

    If $H$, then $B$, $\sim C$, $G$, and $\sim B$.
    This leads to contradiction, thus, $\sim H$

    If $B$, then $\sim C$, $G$, and $\sim B$.
    This leads to contradiction, thus, $\sim B$.

    \phantom{ }

    Solutions:

    \begin{adjustwidth}{2em}{0em}
        $\sim B \land \sim C \land G \land \sim H$

        \hspace*{3.75em}$\text{ OR }$

        $\sim B \land C \land G \land \sim H$

        \hspace*{3.75em}$\text{ OR }$

        $\sim B \land C \land \sim G \land \sim H$
    \end{adjustwidth}

    This can be confirmed by the below truth table, the table
    can be read by looking for rows with all true statements
    made by the detective. These rows correspond to some combination
    of truth values of $B$, $C$, $G$, and $H$:

    $$
    \renewcommand{\arraystretch}{2}
    \begin{array}{|c|c|c|c||c|c|c|c|}
    \hline
    B&C&G&H&\sim B \rightarrow \sim H&\sim (C \land B)&\sim (\sim C \land \sim G)&B \rightarrow \sim G \\
    \hline\hline
    \textsc{t}&\textsc{t}&\textsc{t}&\textsc{t}&\textsc{t}&\textsc{f}&\textsc{t}&\textsc{f} \\
    \hline
    \textsc{t}&\textsc{t}&\textsc{t}&\textsc{f}&\textsc{t}&\textsc{f}&\textsc{t}&\textsc{f} \\
    \hline
    \textsc{t}&\textsc{t}&\textsc{f}&\textsc{t}&\textsc{t}&\textsc{f}&\textsc{t}&\textsc{t} \\
    \hline
    \textsc{t}&\textsc{t}&\textsc{f}&\textsc{f}&\textsc{t}&\textsc{f}&\textsc{t}&\textsc{t} \\
    \hline
    \textsc{t}&\textsc{f}&\textsc{t}&\textsc{t}&\textsc{t}&\textsc{t}&\textsc{t}&\textsc{f} \\
    \hline
    \textsc{t}&\textsc{f}&\textsc{t}&\textsc{f}&\textsc{t}&\textsc{t}&\textsc{t}&\textsc{f} \\
    \hline
    \textsc{t}&\textsc{f}&\textsc{f}&\textsc{t}&\textsc{t}&\textsc{t}&\textsc{f}&\textsc{t} \\
    \hline
    \textsc{t}&\textsc{f}&\textsc{f}&\textsc{f}&\textsc{t}&\textsc{t}&\textsc{f}&\textsc{t} \\
    \hline
    \textsc{f}&\textsc{t}&\textsc{t}&\textsc{t}&\textsc{f}&\textsc{t}&\textsc{t}&\textsc{t} \\
    \hline
    \textsc{f}&\textsc{t}&\textsc{t}&\textsc{f}&\textsc{t}&\textsc{t}&\textsc{t}&\textsc{t} \\
    \hline
    \textsc{f}&\textsc{t}&\textsc{f}&\textsc{t}&\textsc{f}&\textsc{t}&\textsc{t}&\textsc{t} \\
    \hline
    \textsc{f}&\textsc{t}&\textsc{f}&\textsc{f}&\textsc{t}&\textsc{t}&\textsc{t}&\textsc{t} \\
    \hline
    \textsc{f}&\textsc{f}&\textsc{t}&\textsc{t}&\textsc{f}&\textsc{t}&\textsc{t}&\textsc{t} \\
    \hline
    \textsc{f}&\textsc{f}&\textsc{t}&\textsc{f}&\textsc{t}&\textsc{t}&\textsc{t}&\textsc{t} \\
    \hline
    \textsc{f}&\textsc{f}&\textsc{f}&\textsc{t}&\textsc{f}&\textsc{t}&\textsc{f}&\textsc{t} \\
    \hline
    \textsc{f}&\textsc{f}&\textsc{f}&\textsc{f}&\textsc{t}&\textsc{t}&\textsc{f}&\textsc{t} \\
    \hline
    
\end{array}
    $$

    \phantom{ }
\end{Solution}

\vspace*{16em}

\section*{4. }

\begin{Problem}
    (5pts) Prove if $n \in \mathbb Z$, then $n^2+5n-3$ is odd.
\end{Problem}


\begin{Solution}
    Direct Proof:

    \begin{adjustwidth}{2em}{0em}
            

        Either, $n$ is even or odd.

        \phantom{ }

        \textbf{Case 1: }$n$ is even

        \begin{adjustwidth}{2em}{0em}
            $n$ can be expressed as $2|n$, which means that
            $\exists a \in \mathbb Z, 2a = n$.

            \phantom{ }

            Substituting $n$, we get $(2a)^2 + 5(2a) - 3$

            \hspace*{9em}$= 4a^2 + 10a - 3$

            \hspace*{9em}$= 4a^2 + 10a - 4 + 1$

            \hspace*{9em}$= 2(2a^2 + 5a - 2) + 1$

            \phantom{ }

            An odd number is a number of the form $2b + 1$
            for some integer $b$. 
            
            Because $2a^2 + 5a - 2$ is an
            integer, $n^2 + 5n - 3$ is odd. 
        \end{adjustwidth}

        \phantom{ }

        \textbf{Case 2: }$n$ is odd

        \begin{adjustwidth}{2em}{0em}
            $n$ can be expressed as $2|n+1$, which means that
            $\exists a \in \mathbb Z, 2a - 1 = n$.

            \phantom{ }

            Substituting $n$, we get $(2a - 1)^2+5(2a - 1) - 3$

            \hspace*{9em}$= 4a^2-4a+1+10a-5-3$

            \hspace*{9em}$= 4a^2+6a-7$

            \hspace*{9em}$= 4a^2+6a-8+1$

            \hspace*{9em}$=2(2a^2+3a-4)+1$

            \phantom{ }

            An odd number is a number of the form $2b + 1$
            for some integer $b$. 
            
            Because $2a^2+3a-4$ is an
            integer, $n^2 + 5n - 3$ is odd. 
        \end{adjustwidth}

        \phantom{ }

        $n^2 + 5n - 3$ is odd in every case $n \in \mathbb Z$.

        \phantom{ }

        Thus, if $n \in \mathbb Z$, then $n^2 + 5n - 3$ is odd. 
    \end{adjustwidth}
    \phantom{ }
\end{Solution}

\vspace*{17em}

\section*{5. }

\begin{Problem}
    (5pts) Suppose $a, b \in \mathbb Z$ where $\mathbb Z$ denotes all integers. 
    Try to prove the following statement using contraposition:

    If $ab$ is even, then $a$ and $b$ are not both odd.
\end{Problem}

\begin{Solution}
    Proof by Contraposition:

    \begin{adjustwidth}{2em}{0em}
        $a$ and $b$ are both odd, thus $ab$ is odd. 

        \phantom{ }

        Both $a$ and $b$ are of the form $2x + 1$ and $2y + 1$, 
        respectively, where $x,y \in \mathbb Z$. 

        \phantom{ }

        Substituting $a$ and $b$: $ab \implies (2x + 1)(2y + 1)$
        
        \hspace*{12em}$= 4xy + 2x + 2y + 1$ 
        
        \hspace*{12em}$= 2(2xy + x + y) + 1$.

        \phantom{ }

        An odd number is a number of the form $2\lambda + 1$
        for some integer $\lambda$.

        $2xy + x + y$ is an integer, so $ab$ is odd.

        \phantom{ }

        Thus, when $a$ and $b$ are both odd, $ab$ is odd.

    \end{adjustwidth}
    \phantom{ }
\end{Solution}



% % Using environments for block coloring
% \begin{Problem}
% \section*{Problem 2}
% What is the sum of 2 and 3?
% \end{Problem}

% \begin{Solution}
% The sum of 2 and 3 is 5.
% \end{Solution}

\end{document}